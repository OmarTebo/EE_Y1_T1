\documentclass[12pt]{article}

\usepackage{lipsum}
\usepackage{mathtools}

\title{Applications of Second-Order Differnetial Equations}

\begin{document}
  
  \maketitle

  Second-order linear differential equations have a variety of applications in science and
  engineering. In this section we explore two of them: the vibration of springs and
  electric circuits. \\

  \section{Vibrating Springs}
    We consider the motion of an object with mass m at the end of a spring that is either
    vertical (as in Figure 1) or horizontal on a level sirface (as in Figure 2). \\
    \; In Section 6.5 we discussed Hooke's Law, which says that if the spring is stretched (or
        compressed) x units from its natural length, then it exerts a force that is
    proportional to x:\\

    $$ \text{restoring force} = -kx \\ $$

    where $k$ is a positive constant (called the \textbf{spring constant}). if we ignore any
    external resistance forces (due to air resistance or friction) then, by Newton's Second
    Law (force equals mass times acceleration), we have: 

    \begin{equation}
      m\frac{d^2x}{dt^2} = -kx \; \text{or} \; m\frac{d^2x}{dt^2} + kx = 0 
    \end{equation}

    This is a second-order linear differential equation. Its auxiliary equation is $mr^2
    + k = 0$ with roots $r = \pm\omega i$ where $\omega = \sqrt{k/m}$. Thus, the general
    solution is 

      \begin{equation*}
        x(t) = c_1 \cos\omega t + c_2\sin\omega t
      \end{equation*}

      which can also be written as 

      \begin{equation*}
        x(t) = A\cos(\omega t + \delta)
      \end{equation*}

      where 
      \begin{align*}
        \omega &= \sqrt{\frac{k}{m}} & \: \text{(frequency)} \\
        A &= \sqrt{c_1^2 + c_2^2} & \: \text{(amplitude)} \\
        \cos\delta &= \frac{c_1}{A} & \sin\delta = -\frac{c_2}{A} & \: \text{($\delta$ is the phase angle)}
      \end{align*}

      (See Excercise 17.) This type of motion is called \textbf{simple harmonic motion} \\

      \textbf{EXAMPLE 1} A spring with a mass of 2 kg has natural length 0.5 m. A force of
      25.6 N is required to maintain it stretched to a length of 0.7 m. If the spring is
      stretched to a length of 0.7 m and then released with initial velocity 0, find the
      position of the mass at any time $t$. \\

      \textbf{SOLUTION} From Hooke's Law, the force required to stretch the spring is 

      \begin{equation*}
        k(0.2) = 25.6
      \end{equation*}

      so $k = 25.6/0.2 = 128$. Using this value of the spring constant $k$, together with m = 2 in Equation 1, we have 
      
      \begin{equation*}
        2\frac{d^2x}{dt^2} +128x = 0
      \end{equation*}

      As in the earlier general discussion, the solution of this equation is 
      
      \begin{equation}
        x(t) = c_1\cos8t + c_2\sin8t
      \end{equation}

      We are given the initial condition that $x(0) = 0.2$. But, from equation 2, $x(0) = c_1$.
      Therefore, $c_1 = 0.2$. Differentiating Equation 2, we get 
      
      \begin{equation*}
        x'(t) = -8c_1\sin8t + 8c_2\cos8t
      \end{equation*}

      Since the initial velocity is give as $x'(0) = 0$, we have $c_2 = 0$ and so the solution is 

      \begin{equation*}
        x(t) = \frac{1}{5}\cos8t
      \end{equation*}


      \section{Damped Vibration}
      We next consider the motion of a spring that is subject to a frictional force (in the case
          of horizontal spring of Figure 2) or a damping force (in the case where a vertical
            spring moves through a fluid as in Figure 3). An example is the dmaping force
          supplied by a shock absorber in a car or a bicycle. \\
      \quad We assume that the damping force is proportional to the velocity of the mass and acts
      in the direction opposite of the motion. (This has benn confirmed, at least
          approximately, by some physical experiments.) Thus 

      \begin{equation*}
        \text{damping force } = -c\frac{dx}{dt}
      \end{equation*}

      where $c$ is a positive constant, called the \textbf{damping constant.} Thus, in this
      case, Newton's Second Law gives 

      \begin{equation*}
        m\frac{d^2x}{dtx^2} = \text{restoring force} + \text{damping force} = -kx - c\frac{dx}{dt}
      \end{equation*}

      or 
      \begin{equation}
        \boxed
        {
          m\frac{d^2x}{dt^2} + c\frac{dx}{dt} + kx = c
        }
      \end{equation}

      Equation 3 is a second-order linear differential equation and its auxiliary equation
      is $mr^2 + cr + k = 0$. The roots are 

      \begin{equation}
        r_1 = \frac{-c + \sqrt{c^2 - 4mk}}{2m} \; r_2 = \frac{-c - \sqrt{c^2 - 4mk}}{2m}
      \end{equation}

     We need to discuss three cases. \\ \\

     \textbf{CASE I $c^2 - 4mk > 0 $ (overdamping)} \\

     In this case $r_1$ and $r_2$ are distinct real roots and 
      
     \begin{equation*}
      x = c_1e^{r_1t} + c_2e^{r_2t}
     \end{equation*}

      Since $c$, $m$, and $k$ are all positive, we have $\sqrt{c^2 - 4mk} < c$, so the roots
      $r_1$ and $r_2$ given by Equations 4 must both be negative. This shows that $x
      \rightarrow 0$ as $t \rightarrow x$. Typical graphs of $x$ as a function of $t$ are shown
      in Figure 4. Notice that oscillations do not occur. (It's possible for the mass to pass
          through the equilibrium position once, but only once.) This is because $c^2 > 4mk$
      means there is a strong damping force (high-viscosity oil or grase) compared with a
      weak spring or small mass. \\

      \textbf{CASE II $c^2 - 4mk = 0$ (critical damping)} \\
      This case corresponds to equal roots 

      \begin{equation*}
        r_1 = r_2 = -\frac{c}{2m}
      \end{equation*}

      and the solution is given by 
      
      \begin{equation*}
        x = (c_1 + c_2t)e^{-(c/2m)t}
      \end{equation*}

      It is similar to Case I, and typical graphs resemble those in Figure 4 (see Exercise 12),
      but the damping is just sufficient to surpress vibrations. Any decrease in the viscosity
      of the fluid leads to the vibrations of the following case.

      \textbf{CASE III $c^2 - 4mk < 0$ (underdamping)} \\

      Here the roots are complex: 

      \begin{equation*}
        \begin{array} 
          r_1 \\
          r_2 
        \end{array}   = -\frac{c}{2m} \pm \omega i 
      \end{equation*}

\end{document}
